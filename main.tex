% В этом файле следует писать текст работы, разбивая его на
% разделы (section), подразделы (section) и, если нужно,
% главы (chapter).

% Предварительно следует указать необходимую информацию
% в файле SETUP.tex

\input{preamble.tex}
\hyphenation{Monad Applicative Functor Control Data}
    \tikzstyle{every node}=[shape=rectangle, color=black, rounded corners,%
    text=black, anchor=west]
    \tikzstyle{selected}=[shape=rectangle, rounded corners,%
    top color=gray,%
    bottom color=gray, text=white]
    \tikzstyle{optional}=[dashed,fill=gray!50]

\begin{document}

\Intro
\section*{Цель работы}
\section*{Постановка задачи}

\chapter{Предварительные сведения}

\chapter{Алгоритм выведения типов}
\section{Основные понятия}
Для начала введём основные термины, которые будут использоваться при описании алгоритма выведения типов.
\begin{description}
	\item[Компонент.] Компонентами являются встроенные функции. <Добавить также изображение компонента>
	\item[Функция.] Если не указано иное, функцией будем называть определённую пользователем функцию.
	\item[Вызывающий компонент (caller).] Компонент, предназначающийся для использования определённой пользователем функции.
	\item[Связь (connection).]
	\item[Вход (connection point).] Вход компонента соответствует одному аргументу функции. На вход можно подать значение (одно из возможных значений типа, который назначен данному входу) либо другой компонент. Компонент может не иметь входов, если это: 
	\begin{enumerate}[1)]
		\item аргумент определяемой пользователем функции;
		\item вызывающий компонент (caller) пользовательской функции без аргументов.
	\end{enumerate} 
	\item[Контекст.] description
	\item[Инферер (inferer).] description
	\item[Алиас (alias).] description
\end{description}
 
\chapter{Проблемы и их решения}

\chapter{Детали реализации}

\chapter{Кодогенерация}

\chapter{Тестирование}
\section{Тест <<Переприсоединение>>}
Тест <<Переприсоединение>> (англ. \textit{test-reconnect}) был задумал с целью автоматизации проверки того, что состояние инферера каждого компонента остаётся прежним после удаления и восстановления одной из входящих связей.
\section{Тест <<Все ко всем>>}
Тест <<Все ко всем>> (англ. \textit{all-to-all test}) воспроизводит попытки создания всевозможных связей для каждого конкретного примера. Это необходимо для проверки того, что все ошибочные ситуации обрабатываются правильно. К ошибочным ситуациям относятся:
\begin{itemize}
	\item попытка создания связи, которая приводит к циклу в синтаксическом дереве;
	\item попытка создания связи, которая приводит к появлению бесконечного типа;
\end{itemize}


% Печать списка литературы (библиографии)
\printbibliography[%{}
    heading=bibintoc%
    %,title=Библиография % если хочется это слово
]
% Файл со списком литературы: biblio.bib
% Подробно по оформлению библиографии:
% см. документацию к пакету biblatex-gost
% http://ctan.mirrorcatalogs.com/macros/latex/exptl/biblatex-contrib/biblatex-gost/doc/biblatex-gost.pdf
% и огромное количество примеров там же:
% http://mirror.macomnet.net/pub/CTAN/macros/latex/contrib/biblatex-contrib/biblatex-gost/doc/biblatex-gost-examples.pdf

%\appendix
%\ifthenelse{\value{worktype} > 1}{%
  %\addtocontents{toc}{%
      %\protect\renewcommand{\protect\cftchappresnum}{\appendixname\space}%
      %\protect\addtolength{\protect\cftchapnumwidth}{\widthof{\appendixname\space{}} - \widthof{Глава }}%
  %}%
%}{
  %\addtocontents{toc}{%
      %\protect\renewcommand{\protect\cftsecpresnum}{\appendixname\space}%
      %\protect\addtolength{\protect\cftsecnumwidth}{\widthof{\appendixname\space{}}}%
  %}%
%}

%\section{Пример работы программы}

%Здесь длинный листинг с примером работы.

\end{document}
